%%Nevynechávat volný řádek

Laser shock peening is a surface treatment process used to improve the physico-chemical characteristics (fatigue life, corrosion resistance) of metallic components. Laser shock peening induces residual stresses beneath the treated surface of metallic materials. Laser shock peening applications have been on the rise in recent years, mainly due to the ever-increasing energy and decreasing prices of laser systems with parameters suitable for this treatment process. This graduation thesis seeks to solve the following problem: how can RoboDK with its Python application user interface be effectively used to generate robotic arm programs for laser shock peening applications. The problem is solved by modifying an existing RoboDK post processor. The created solution provides the possibility of generating robotic arm programs specially tailored for laser shock peening applications. One of the main contributions of this graduation thesis is to simplify the generation of robotic arm programs for parts with complex geometries.


